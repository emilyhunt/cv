% --------------------------
% Main source file for Emily L. Hunt's CV.
% --------------------------

\documentclass[12pt, letterpaper]{hunt-cv}

% ---- Settings ----
\renewcommand{\cvDate}{\today}
\renewcommand{\cvName}{Emily L. Hunt}  % Your name! As displayed in the CV
\renewcommand{\cvPublicationName}{Hunt, Emily L.}  % Name to highlight in author lists
\renewcommand{\cvEmail}{emily.hunt.physics@gmail.com}
\renewcommand{\cvWebsite}{emily.space}
\renewcommand{\cvGitHub}{github.com/emilyhunt}
\renewcommand{\cvADS}{ui.adsabs.harvard.edu/search/q=orcid\%3A0000-0002-5555-8058&sort=date\%20desc\%2C\%20bibcode\%20desc&p_=0}

% PDF metadata
\hypersetup{
  colorlinks = true,
  pdfkeywords = {astrophysics, astronomy, physics},
  pdftitle = {Curriculum Vitae},
  pdfsubject = {Curriculum Vitae},
  pdfpagemode = UseNone,
  linkcolor = emilyred,
  urlcolor = emilyred
}

% ---- Body ----
\begin{document}
\thispagestyle{plain}  % Removes fancyhdr on first page

\cvTitle

% ---------------------------------------
\section*{Research Profile}

Astronomer with interests in machine learning and statistics. Highly skilled programmer with 10+ years of programming experience. During my Ph.D., I used Gaia data and various machine learning techniques to create the largest ever catalogue of star clusters in the Milky Way. I am looking to work on exciting and challenging data analysis projects.


% ---------------------------------------
% \section*{Employment}

% % \begin{itemize}
% %     \item 2023 -- 2024, Postdoc, Heidelberg University
% % \end{itemize}

% \vspace{0.8cm}


% ---------------------------------------
\section*{Education \& Employment}

\subsection*{Postdoctoral researcher, Heidelberg University \hfill 2023 -- 2024}\vspace{0.3cm}

\subsection*{Ph.D. in Astronomy \hfill 2023}
\begin{addmargin}[1em]{0em}
    \textbf{Heidelberg University, Germany} ~ (IMPRS-HD Graduate School)\\
    \textbf{Thesis:} ``Improving the census of open clusters in the Milky Way with data from Gaia''\\
    \textbf{Advisor:} S. Reffert
\end{addmargin}

\subsection*{M.Phys. Physics with Astronomy \hfill 2019}
\begin{addmargin}[1em]{0em}
    \textbf{University of Bath, United Kingdom}\\
    \textbf{Thesis:} ``Inference of photometric galaxy redshifts with a
    mixture density network''\\
    \textbf{Advisor:} S. Wuyts
\end{addmargin}

% \vspace{0.35cm}
% In addition, I completed a research internship while at the University of Bath with V. Scowcroft, using a Bayesian method to de-redden Cepheid variable stars.

% \begin{itemize}
%     \item Ph.D. 2023. International Max Planck Research School for Astronomy \& Cosmic Physics at the University of Heidelberg. \emph{Advisor: S. Reffert}
%     \item M.Phys. 2019. University of Bath. \emph{Advisor: S. Wuyts}
%     \item Research internship, 2018. University of Bath. \emph{Advisor: V. Scowcroft}
% \end{itemize}


% ---------------------------------------
\section*{Publications}

\href{\cvADSLink}{ADS search \faLink}


\begin{etaremune}
    \item \publication
        {Hunt, Emily L. and Reffert, Sabine}
        {in prep.}
        {Improving the open cluster census. III. The masses and dynamics of open clusters in the Milky Way}
    \item \publication
        {Hunt, Emily L. and Reffert, Sabine}
        {2023}
        {Improving the open cluster census. II. An all-sky cluster catalogue with Gaia DR3}
        [A\&A, 673, A114]
        [https://ui.adsabs.harvard.edu/abs/2023A\%26A...673A.114H]
        [12]
    \item \publication
        {Hunt, Emily L. and Reffert, Sabine}
        {2021}
        {Improving the open cluster census. I. Comparison of Clustering Algorithms applied to Gaia DR2 Data}
        [A\&A, 646, A104]
        [https://ui.adsabs.harvard.edu/abs/2021A\%26A...646A.104H]
        [57]
\end{etaremune}


% ---------------------------------------
\section*{Selected Presentations}

\begin{itemize}
    \item \textbf{Talk,} .Astronomy 12 -- Flatiron Institute, New York, NY, USA \hfill (upcoming) 2023
    \item \textbf{Colloquium,} Königstuhl Colloquium -- MPIA, Heidelberg, Germany \hfill (upcoming) 2023
    \item \textbf{Talk,} National Astronomy Meeting -- Coventry, England, UK \hfill 2022
    \item \textbf{Invited talk,} EAS (SS34) -- Valencia, Spain \hfill 2022
    \item \textbf{Talk,} EAS (SS24) -- Valencia, Spain \hfill 2022
    \item \textbf{Talk,} EAS (SS15) -- Valencia, Spain \hfill 2022
    \item \textbf{Talk,} LGBTQ+ STEMinar  -- University of Glasgow, Scotland, UK \hfill 2022
    \item \textbf{Seminar,} Galaxy group -- ARI, Heidelberg, Germany \hfill 2021
    \item \textbf{Seminar,} Astronomy group -- University of Hertfordshire, England, UK \hfill 2021
    \item \textbf{Talk,} Star Clusters: The Gaia Revolution \hfill 2021
    \item \textbf{Invited talk,} EAS (S32) -- Leiden, Netherlands \hfill 2021
    \item \textbf{Talk,} EAS (S15) -- Leiden, Netherlands \hfill 2021
    \item \textbf{Seminar,} SFB 881 -- Heidelberg, Germany \hfill 2021
    \item \textbf{Seminar,} Gaia group -- University of Vienna, Austria \hfill 2021
    \item \textbf{Seminar,} Astronomy group -- University of Bath, England, UK \hfill 2020
    \item \textbf{Seminar,} Milky Way group -- MPIA, Heidelberg, Germany \hfill 2020
\end{itemize}

% ---------------------------------------
% \section*{Selected Outreach}

% \begin{itemize}
%     \item \emph{Searching for clusters of stars.} OUTer SPACE, Max Planck Institute for Astronomy, 2023.*\cross
% \end{itemize}

% ---------------------------------------
\section*{Workshops Attended}

\begin{itemize}
    \item \textbf{From star clusters to field populations} -- Florence, Italy \hfill (upcoming) 2023
    \item \textbf{.Astronomy 12} -- Flatiron institute, New York, NY, USA \hfill (upcoming) 2023
    \item \textbf{CZS school on Scientific Machine Learning} -- Heidelberg, Germany \hfill 2023
    \item \textbf{GaiaUnlimited Community Workshop} -- Heidelberg, Germany \hfill 2022
    \item \textbf{..Astronomy} -- online \hfill 2020
\end{itemize}


% ---------------------------------------
\section*{Press}

\begin{itemize}
    \item \textbf{Space.com article contribution} -- journal name change policies \hfill 2021
    \item \textbf{Thrillist.com article contribution} -- LGBTQ+ outreach \hfill 2020
    \item \textbf{Neue Zürcher Zeitung (NZZ) radio interview} -- about Gaia EDR3 \hfill 2020
\end{itemize}


% ---------------------------------------
\section*{Awards}

\begin{itemize}
    \item \textbf{University of Bath IMI Undergraduate Research Internship} -- £2000 \hfill 2018\\
\end{itemize}


% ---------------------------------------
\section*{Teaching \& Supervision}

\begin{itemize}
    \item \textbf{Astronomy Lab Course,} Heidelberg University \hfill 2021
    \item \textbf{Introduction to Astronomy I,} Heidelberg University \hfill 2020
    \item \textbf{Co-supervisor of MSc student,} Heidelberg University \hfill 2020-2021
\end{itemize}


% ---------------------------------------
\section*{Community Service}

\subsection*{Conferences and workshops}

\begin{itemize}
    \item \textbf{SOC} for .Astronomy 12 \hfill 2023
    \item \textbf{Project leader} at CZS school on Scientific Machine Learning in Astrophysics \hfill 2023
    \item \textbf{Session leader} at GaiaUnlimited (open cluster selection functions) \hfill 2022
\end{itemize}

\subsection*{Open-source software \href{\cvGitHubLink}{\faGithub}}

\begin{itemize}
    \item \textbf{Bluesky Astronomy feeds} -- lead developer of \href{https://github.com/emilyhunt/bluesky-astronomy-feeds}{astronomy community feeds} on Bluesky social network, which are used daily by hundreds of astronomers to interact
    \item \textbf{ocelot} -- lead developer of an upcoming open cluster analysis Python package
\end{itemize}


% \subsubsection*{Reviewing}

% \begin{itemize}
%     \item 
% \end{itemize}


% ---------------------------------------
\section*{Relevant expertise}

\subsection*{Programming languages}

\begin{itemize}
    \item \textbf{Python:} expert (e.g. numpy, tensorflow, emcee)
    \item \textbf{C/C++:} intermediate
    \item \textbf{JavaScript:} intermediate (Svelte, SvelteKit)
    \item \textbf{Java:} basic
\end{itemize}

\subsection*{Tools and scripting languages}

\begin{itemize}
    \item \textbf{Git/GitHub:} expert
    \item \textbf{LaTeX:} expert
    \item \textbf{HTML/CSS:} intermediate
    \item \textbf{ADQL/SQL:} basic
\end{itemize}

\subsection*{Languages}

\begin{itemize}
    \item \textbf{English:} native speaker
    \item \textbf{German:} intermediate
\end{itemize}


\end{document}