% --------------------------
% Main source file for Emily L. Hunt's CV.
% --------------------------

\documentclass[12pt, letterpaper]{hunt-cv}

\begin{document}
\cvTitle

\section*{Personal Information}

\begin{itemize}
    \item Nationality: British
    \item Sex: female
    \item Marital status: single
\end{itemize}

% \section*{Research statement}

% I am an ambit

% ---------------------------------------
\section*{Education}

\begin{itemize}
    \item Ph. D. 2023. International Max Planck Research School for Astronomy \& Cosmic Physics at the University of Heidelberg. \emph{Advisor: S. Reffert}
    \item M. Phys. 2019. University of Bath. \emph{Advisor: S. Wuyts}
    \item Research internship, 2018. University of Bath. \emph{Advisor: V. Scowcroft}
\end{itemize}


% ---------------------------------------
\section*{Publications}

\begin{itemize}
    \item \textbf{Emily L. Hunt} and Sabine Reffert (2023). ``Improving the open cluster census. III. The masses and dynamics of open clusters in the Milky Way''. \emph{In prep.}
    \item \textbf{Emily L. Hunt} and Sabine Reffert (2023). ``Improving the open cluster census. II. An all-sky cluster catalogue with Gaia DR3''. A\&A (accepted March 21\textsuperscript{st}, 2023).
    \item \textbf{Emily L. Hunt} and Sabine Reffert (2021). ``Improving the open cluster census. I. Comparison of Clustering Algorithms applied to Gaia DR2 Data''. A\&A 646, A104.
\end{itemize}


% ---------------------------------------
\section*{Selected Presentations}

\begin{itemize}
    \item \emph{Searching for clusters of stars.} OUTer SPACE, Max Planck Institute for Astronomy, 2023.*\cross
    \item \emph{The power (and caveats) of clustering algorithms with examples from use on Gaia data.} NAM 2022 (Techniques 2).
    \item \emph{The open cluster renaissance has only just begun: Exciting new insights from an all-sky Gaia EDR3 cluster census.} EAS 2022 (SS15).
    \item \emph{Name change policies in astronomy journals: How they were achieved and lessons we can learn.} EAS 2022 (SS34).*
    \item \emph{Approximate Bayesian neural networks with `Flipout' weight perturbations.} EAS 2022 (SS24).
    \item \emph{Ancestry in space: Looking for families of stars with machine learning.} LGBTQ+ STEMinar, 2022.\cross
    \item \emph{An all-sky open cluster census with Gaia EDR3.} Galaxy group seminar, Astronomisches Rechen Institut of Heidelberg University, 2021.*
    \item \emph{An all-sky open cluster census with Gaia EDR3.} Astronomy seminar, University of Hertfordshire, 2021.*
    \item \emph{A more complete and accurate open cluster census with Gaia EDR3.} Star Clusters: The Gaia Revolution, 2021.
    \item \emph{Uncertainty in machine learning: are Bayesian neural networks viable in 2021?} EAS 2021 (S32).*
    \item \emph{A more complete and accurate open cluster census with Gaia EDR3.} EAS 2021 (S15).
    \item \emph{Searching for open clusters with Gaia.} SFB 881 seminar, Center for Astronomy of Heidelberg University, 2021.
    \item \emph{Searching for open clusters with Gaia.} Gaia group seminar, University of Vienna, 2021.*
    \item \emph{Comparing methods to search for open clusters with Gaia.} Astronomy seminar, University of Bath, 2020.*
    \item \emph{Comparing methods to search for new open clusters with Gaia.} Milky Way group seminar, Max Planck Institute for Astronomy, 2020.*
\end{itemize}

* = invited; \cross = outreach.
\vspace{0.5cm}

% ---------------------------------------
\section*{Workshops}

\begin{itemize}
    \item CZS summer school on Scientific Machine Learning in Astrophysics, Heidelberg, 2023.
    \item GaiaUnlimited Community Workshop, Heidelberg, 2022. (Led session on open cluster selection functions.)
    \item dotdotAstronomy (2020).
\end{itemize}


% ---------------------------------------
\section*{Awards}

\begin{itemize}
    \item University of Bath IMI Undergraduate Research Internship (funded), 2018
\end{itemize}


% ---------------------------------------
\section*{Teaching}

\begin{itemize}
    \item Astronomy Lab Course, Heidelberg University, 2021
    \item Introduction to Astronomy I, Heidelberg University, 2020
\end{itemize}


% ---------------------------------------
\section*{Relevant expertise}

\subsection*{Programming languages}

\begin{itemize}
    \item \textbf{Python:} expert, \emph{with modules including: numpy, matplotlib, tensorflow, and emcee.}
    \item \textbf{C/C++:} intermediate
    \item \textbf{JavaScript:} intermediate, \emph{using the SvelteKit framework.}
    \item \textbf{Java:} basic
\end{itemize}

\subsection*{Tools and scripting languages}

\begin{itemize}
    \item \textbf{Git/GitHub:} expert
    \item \textbf{LaTeX:} expert
    \item \textbf{ADQL/SQL:} basic
    \item \textbf{HTML/CSS:} basic
\end{itemize}


\end{document}