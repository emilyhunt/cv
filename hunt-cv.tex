% --------------------------
% Main source file for Emily L. Hunt's CV.
% --------------------------

\documentclass[12pt, letterpaper]{hunt-cv}

% ---- Settings ----
\renewcommand{\cvDate}{\today}
\renewcommand{\cvName}{Emily L. Hunt}  % Your name! As displayed in the CV
\renewcommand{\cvPublicationName}{Hunt, Emily L.}  % Name to highlight in author lists
\renewcommand{\cvEmail}{emily.hunt.physics@gmail.com}
\renewcommand{\cvWebsite}{emily.space}
\renewcommand{\cvGitHub}{github.com/emilyhunt}
\renewcommand{\cvADS}{ui.adsabs.harvard.edu/search/q=orcid\%3A0000-0002-5555-8058&sort=date\%20desc\%2C\%20bibcode\%20desc&p_=0}

% PDF metadata
\hypersetup{
  colorlinks = true,
  pdfkeywords = {astrophysics, astronomy, physics},
  pdftitle = {Curriculum Vitae},
  pdfsubject = {Curriculum Vitae},
  pdfpagemode = UseNone,
  linkcolor = emilyred,
  urlcolor = emilyred
}

% ---- Body ----
\begin{document}
\thispagestyle{plain}  % Removes fancyhdr on first page

\cvTitle

% ---------------------------------------
\section*{Research Profile}

Astronomer with interests in machine learning and statistics. Highly skilled programmer with 10+ years of programming experience. During my Ph.D., I used Gaia data and various machine learning techniques to create the largest ever catalogue of star clusters in the Milky Way. I am looking to work on exciting and challenging data analysis projects.


% ---------------------------------------
% \section*{Employment}

% % \begin{itemize}
% %     \item 2023 -- 2024, Postdoc, Heidelberg University
% % \end{itemize}

% \vspace{0.8cm}


% ---------------------------------------
\section*{Education \& Employment}

\subsection*{2023 -- 2024: Postdoctoral researcher, Heidelberg University}\vspace{0.3cm}

\subsection*{2023 -- Ph.D. in Astronomy}
\begin{addmargin}[1em]{0em}
    \textbf{Heidelberg University, Germany} ~ (IMPRS-HD Graduate School)\\
    \textbf{Thesis:} ``Improving the census of open clusters in the Milky Way with data from \emph{Gaia}''\\
    \textbf{Advisor:} S. Reffert
\end{addmargin}

\subsection*{2019 -- M.Phys. Physics with Astronomy}
\begin{addmargin}[1em]{0em}
    \textbf{University of Bath, United Kingdom}\\
    \textbf{Thesis:} ``Inference of photometric galaxy redshifts with a
    mixture density network''\\
    \textbf{Advisor:} S. Wuyts
\end{addmargin}

% \vspace{0.35cm}
% In addition, I completed a research internship while at the University of Bath with V. Scowcroft, using a Bayesian method to de-redden Cepheid variable stars.

% \begin{itemize}
%     \item Ph.D. 2023. International Max Planck Research School for Astronomy \& Cosmic Physics at the University of Heidelberg. \emph{Advisor: S. Reffert}
%     \item M.Phys. 2019. University of Bath. \emph{Advisor: S. Wuyts}
%     \item Research internship, 2018. University of Bath. \emph{Advisor: V. Scowcroft}
% \end{itemize}


% ---------------------------------------
\section*{Publications}

\href{\cvADSLink}{ADS search \faLink}


\begin{etaremune}
    \item \publication
        {Hunt, Emily L. and Reffert, Sabine}
        {in prep.}
        {Improving the open cluster census. III. The masses and dynamics of open clusters in the Milky Way}
    \item \publication
        {Hunt, Emily L. and Reffert, Sabine}
        {2023}
        {Improving the open cluster census. II. An all-sky cluster catalogue with Gaia DR3}
        [A\&A, 673, A114]
        [https://ui.adsabs.harvard.edu/abs/2023A\%26A...673A.114H]
    \item \publication
        {Hunt, Emily L. and Reffert, Sabine}
        {2021}
        {Improving the open cluster census. I. Comparison of Clustering Algorithms applied to Gaia DR2 Data}
        [A\&A, 646, A104]
        [https://ui.adsabs.harvard.edu/abs/2021A\%26A...646A.104H]
\end{etaremune}


% ---------------------------------------
\section*{Selected Presentations}

\begin{itemize}
    %\item \emph{Searching for clusters of stars.} OUTer SPACE, Max Planck Institute for Astronomy, 2023.*\cross
    \item NAM 2022 (Techniques 2) -- The power (and caveats) of clustering algorithms with examples from use on Gaia data.
    \item EAS 2022 (SS15) -- The open cluster renaissance has only just begun: Exciting new insights from an all-sky Gaia EDR3 cluster census.
    \item EAS 2022 (SS34)* -- Name change policies in astronomy journals: How they were achieved and lessons we can learn. 
    \item EAS 2022 (SS24) -- Approximate Bayesian neural networks with `Flipout' weight perturbations.
    \item LGBTQ+ STEMinar, 2022\cross -- Ancestry in space: Looking for families of stars with machine learning.
    %\item \emph{An all-sky open cluster census with Gaia EDR3.} Galaxy group seminar, Astronomisches Rechen Institut of Heidelberg University, 2021.*
    \item Astronomy seminar, University of Hertfordshire, 2021* -- An all-sky open cluster census with Gaia EDR3. 
    \item Star Clusters: The Gaia Revolution, 2021 -- A more complete and accurate open cluster census with Gaia EDR3.
    \item EAS 2021 (S32)* -- Uncertainty in machine learning: are Bayesian neural networks viable in 2021?
    \item EAS 2021 (S15) -- A more complete and accurate open cluster census with Gaia EDR3.
    %\item \emph{Searching for open clusters with Gaia.} SFB 881 seminar, Center for Astronomy of Heidelberg University, 2021.
    \item Gaia group seminar, University of Vienna, 2021* -- Searching for open clusters with Gaia.
    %\item Astronomy seminar, University of Bath, 2020* -- Comparing methods to search for open clusters with Gaia.
    %\item \emph{Comparing methods to search for new open clusters with Gaia.} Milky Way group seminar, Max Planck Institute for Astronomy, 2020.*
\end{itemize}

* = invited; \cross = outreach.

% ---------------------------------------
\section*{Workshops Attended}

\begin{itemize}
    \item CZS school on Scientific Machine Learning in Astrophysics, Heidelberg, 2023
    \item GaiaUnlimited Community Workshop, Heidelberg, 2022
    \item dotdotAstronomy, 2022
\end{itemize}


% ---------------------------------------
\section*{Awards}

\begin{itemize}
    \item \textbf{University of Bath IMI Undergraduate Research Internship (funded), 2018}\\
    Project: De-reddening Cepheid variable stars with a Bayesian inference method\\
    Advisor: V. Scowcroft\\
    Funding: £2000
\end{itemize}


% ---------------------------------------
\section*{Teaching \& Supervision}

\begin{itemize}
    \item Astronomy Lab Course, Heidelberg University, 2021
    \item Introduction to Astronomy I, Heidelberg University, 2020
    \item Co-supervisor of MSc student, 2020-2021
\end{itemize}


% ---------------------------------------
\section*{Community Service}

\subsection*{Conferences and workshops}

\begin{itemize}
    \item SOC for .Astronomy 12, New York (2023)
    \item Led session on open cluster selection functions at GaiaUnlimited 2022
\end{itemize}

\subsection*{Open-source software \href{\cvGitHubLink}{\faGithub}}

\begin{itemize}
    \item Developing a new open cluster analysis Python package for publication in 2023
    \item Developer of \href{https://github.com/emilyhunt/bluesky-astronomy-feeds}{astronomy community feeds} on Bluesky social network
\end{itemize}


% \subsubsection*{Reviewing}

% \begin{itemize}
%     \item 
% \end{itemize}


% ---------------------------------------
\section*{Relevant expertise}

\subsection*{Programming languages}

\begin{itemize}
    \item \textbf{Python:} expert (e.g. numpy, tensorflow, emcee)
    \item \textbf{C/C++:} intermediate
    \item \textbf{JavaScript:} intermediate (Svelte, SvelteKit)
    \item \textbf{Java:} basic
\end{itemize}

\subsection*{Tools and scripting languages}

\begin{itemize}
    \item \textbf{Git/GitHub:} expert
    \item \textbf{LaTeX:} expert
    \item \textbf{HTML/CSS:} intermediate
    \item \textbf{ADQL/SQL:} basic
\end{itemize}

\subsection*{Languages}

\begin{itemize}
    \item \textbf{English:} native speaker
    \item \textbf{German:} intermediate
\end{itemize}


\end{document}